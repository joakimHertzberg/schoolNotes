\documentclass[12pt]{article}
\usepackage{amsmath}
\usepackage{amssymb}
\usepackage{tikz}
\usepackage{circuitikz}
\usepackage{tkz-euclide}
\usepackage{pgfplots}
\usepackage[many]{tcolorbox}
\pgfplotsset{compat=1.18}
\newtcolorbox{BOX}{
colframe=white,
colback=white,
enhanced,
boxrule=1.5pt,
borderline={0.75pt}{0pt}{dashed},
}
\tikzset{european}
\title{Electronics \& Magnetism}
\author{Hertzberg, Joakim D.}
\date{\today}

%END OF PREAMBLE
\begin{document}
\begin{titlepage}
\clearpage \maketitle
\begin{center}
	FYS01a: Physics 1a
\end{center}
\thispagestyle{empty}
\end{titlepage}

\tableofcontents

\newpage

\section{Fundamental Quanitites}


\subsection{Charge}
Charge is measured in Coulomb ($C$). In equations, it's 
often symbolised with $q_n$.


\subsection{Potential Difference}
\emph{Potential Difference} (alterenatively \emph{Voltage}) is the difference in the amount of energy that a charge carrier has between 2 points. 

\subsection{Flow}
Flow (alterenatively \emph{Amperage}), measured in 
$A$ (\emph{Amperes}), but in equations denoted as $I$ is 
the amount of charge moving through a certain 
cross-section per unit of time. $$I = C \ s^{-1}$$

\newpage

\section{Circuits \& Quanitites}
The structure of a circuit affects how the \emph{Voltage} 
\& \emph{Amperage} behaves. 

\subsection{Circuits in series}

\begin{center}
\begin{circuitikz}[scale=3.0]
\draw (0,0) -- (0,1) to[R=$R_1$] (1,1) to[R=$R_2$] (2,1) -- (2,0) to[battery1] (0,0);

\node[anchor=south east] at(1,1){$V_1$};

\node[anchor=south east] at(2,1){$V_2$};

\node[anchor=north east] at(1,1){$I_1$};

\node[anchor=north east] at(2,1){$I_2$};



\end{circuitikz}
\end{center}


\subsection{Circuits in parallel}
\begin{center}
\begin{circuitikz}[scale=3.0]
\draw (0,0) -- (0,2) to[R=$R_1$] (2,2) -- (2,0) to[battery1] (0,0);
\draw (0,1) to[R=$R_2$, *-*] (2,1);

\node[anchor=south east] at(0.8,0){$V_0$};

\node[anchor=south east] at(0.8,1){$V_2$};

\node[anchor=south east] at(0.8,2){$V_1$};

\node[anchor=north east] at(0.8,0){$I_0$};

\node[anchor=north east] at(0.8,1){$I_2$};

\node[anchor=north east] at(0.8,2){$I_1$};

\end{circuitikz}
\end{center}



\end{document}
