\documentclass[12pt]{article}
\usepackage{amsmath}
\usepackage{bm}
\usepackage{amssymb}
\usepackage{tikz}
\usepackage{tkz-euclide}
\usepackage{circuitikz}
\usepackage[many]{tcolorbox}

\newtcolorbox{boxA}{
colframe=white,
colback=white,
enhanced,
boxrule=1.5pt,
borderline = {0.75pt}{0pt}{dashed}
}

 
\tikzset{european}
\title{Nuclear Physics}
\author{Hertzberg, Joakim D.}
\date{\today}
\begin{document}

%END OF PREAMBLE

%title page
\begin{titlepage}
\maketitle
\begin{center}
\thispagestyle{empty}
\end{center}
\end{titlepage}

%table of contents
\tableofcontents

\newpage


%section 1: Particles
\section{Particles and their properties}
\subsection{The atomic mass unit}

The atomic mass unit (also \emph{AMU}) is defined as $1.66053907 10^{-27} \ Kg$. It has the unit \emph{u}.
It was defined such that the mass of a proton (and a neutron, which has the same mass) is $1u$.

\subsection{The electron}

The electron is an \emph{elementary particle}, which is a particle that has no building blocks, but is itself among the smallest possible building blocks.
The mass of an electron is $0.000548579909 \ u$, and it has a charge of $1.60217663 \times 10^{-19} \ C$, which is equal to the \emph{elementary charge} $e$.

\subsection{The proton}
The proton is a \emph{non-elementary} particle, which consists of \emph{quarks} which are elementary particles. It has a mass of $1.00727647 \ u$, and is positively charged by $+1 \ e$.

\subsection{The neutron}
The neutron is, as it's name implies, neutrally charged, meaning it has a charge of $0$, it has a mass of $1.008664915 \ u$.

\subsection{The positron}

The \emph{positron} is something which may also be encountered, and can easily be confused with the proton. It is significant to note that the positron is an \emph{elementary particle}, and is the anti-matter form of the \emph{electron}, so it has the same mass ($0.000548579909 \ u$), and a charge of $+1 \ e$.





%radiation

\section{Types of radiation}


%alpha-radiation
\subsection{$\bm{\alpha}$-radiation}
\emph{Alpha Radiation} or \emph{Alpha Decay} is when a particle releases another particle, a so-called \emph{$\alpha$-particle} as it decays.
$$^{A}_{Z}X \rightarrow \ \ ^{A-4}_{Z-2}X' + \ ^{4}_{2}\alpha$$

%Q-box
\begin{boxA}
	\textbf{\underline{What \emph{is} an $\bm{\alpha}$-particle?}} \bigbreak
	An alpha-particle looks exactly like a $He$-nucleus, that is, two protons \& 2 neutrons. Ignoring the amount of electrons, following is true: $$^{4}_{2}\alpha \ = \ ^{4}_{2}He$$
\end{boxA}

%beta-radiation
\subsection{$\bm{\beta}$-radiation}
\emph{Beta Radiation} or \emph{Beta Decay} is when a particle releases an electron and a neutrino as it decays, by a neutron splitting into an electron and a proton. 
\bigbreak
$$^{A}_{Z}X \rightarrow \ \ ^{A}_{Z+1}X' \ + \ ^{\ 0}_{-1}e \ + \ \bar{v}$$

%sidenote
\begin{boxA}
	\textbf{\underline{Sidenote:}} \bigbreak
Since elements release electrons as they decay, and gain a proton, the element itself changes to the next element over in the periodic table.
\end{boxA}

\subsection{$\bm{\gamma}$-radiation}

\newpage

%REACTIONS
\section{Nuclear Reactions}

%released energy
\subsection{Energy released}

In nuclear reactions, there is often a \emph{mass deficit} ($\Delta m$), which is then related to a certain energy which that mass converts into, $\Delta E$. Said energy is provided by Albert Einstein's famos equation: $$E = mc^2$$ \\
Which can be translated into: $$\Delta E = \Delta mc^2$$

%binding energy
\subsection{Binding Energy}
It has been found by scientists that there is more mass in an atom than the sum of all its \emph{nucleons}\footnote{A \emph{nucleon} is either a \emph{proton} or \emph{neutron}, i.e. a component of the nucleus}, this mass represents the \emph{binding energy}, which is the energy that is contained by the bonds which hold the neutrons particles together in the nucleus. This means that we can conclude following for an element $^{A}_{Z}X$: $$m_X - \Delta m \ = (Z)m_p + (A-Z)m_n$$
\bigbreak
NOTE: Binding energy is almost always the \emph{energy released} (see \textbf{\emph{3.1}})

\newpage

%example box
\begin{boxA}
	\textbf{\underline{Example:}}\bigbreak
	$^{54}Fe$ has a mass of $53.9396082 \ u$, a proton (with accompanying electron)
 has a mass of $1.0078250319 \ u$, and a neutron has a mass of $1.0086649 \ u$. \bigbreak

 \textbf{Q}: Find the binding energy of iron-54 \bigbreak

 Let $E_b$ be \emph{binding energy}

$$m_{Fe} - \Delta m \ = \ 53.9396 \ u \ = \ 26(m_p) + (54-26)(m_n)$$
$$53.9396082 - \Delta m \ = \ 26(1.0078250319) + 28(1.0086649)$$
$$53.9396082 - \Delta m \ = \ 52.42873823$$
$$\Delta m \ = \ 53.9396082 - 52.42873823$$
$$\Delta m \ = \ 1.51086997 \ u$$ \bigbreak
$$\because \left(\Delta E = \Delta m \times c^2 \right) \land \left(\Delta E \ = \ E_b \right)$$
$$E_b \ = \ \Delta m \times c^2$$
\begin{center}
Convert $m$ from $AMU$ to $Kg$. 
\end{center}
$$m = 1.51086997 \times 1.66053907 \times 10^{-17} \ = \ 2.50885861 \times 10^{-17}$$
$$E_b \ = \ 2.50885861 \times 10^{-17}(299 792 458)^2$$
$$E_b \ = \ 2.254849673 \ J$$
\end{boxA}

\newpage

%types of reactions
\subsection{Reaction Types}

\subsubsection{Transmutation}

It is possible to transmute one particle into the other by fusion. For example: $$^{14}_{\ 7}N \ + \ ^{4}_{2}He \ \rightarrow \ \ ^{17}_{\ 8}O \ + \ ^{1}_{1}H$$

\subsubsection{Fission}
A fission reaction is one where an atom is split into other atoms, this is usually done synthetically by bombarding the original atom with neutrons. $$^{A}_{Z}X \ + \ ^0_1n \ \rightarrow \ ^{A_1}_{Z_{1}}Y \ + \ ^{A-A_1}_{Z-Z_1-1}X' \ + \ 2\left(^{0}_{1}n \right)$$

\subsubsection{Fusion}
...to be continued

\end{document}
