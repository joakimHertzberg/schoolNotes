\documentclass[12pt]{article}
\usepackage{amsmath}
\usepackage{bm}
\usepackage{amssymb}
\usepackage{tikz}
\usepackage{tkz-euclide}
\usepackage{circuitikz}
\usepackage[many]{tcolorbox}

\newtcolorbox{boxA}{
colframe=white,
colback=white,
enhanced,
boxrule=1.5pt,
borderline = {0.75pt}{0pt}{dashed}
}

\tikzset{european}
\title{Nuclear Physics}
\author{Hertzberg, Joakim D.}
\date{\today}
\begin{document}
\begin{titlepage}
\maketitle
\begin{center}
\thispagestyle{empty}
\end{center}
\end{titlepage}
\section{Types of radiation}
\subsection{$\bm{\alpha}$-radiation}
\emph{Alpha Radiation} or \emph{Alpha Decay} is when a particle releases another particle as it decays.
$$^{A}_{Z}X \rightarrow \ \ ^{A-4}_{Z-2}X' + \ ^{4}_{2}a$$

\subsection{$\bm{\beta}$-radiation}
\emph{Beta Radiation} or \emph{Beta Decay} is when a particle releases an electron and a neutrino as it decays, by a neutron splitting into an electron and a proton. 
\bigbreak
$$^{A}_{Z}X \rightarrow \ \ ^{A}_{Z+1}X' \ + \ ^{\ 0}_{-1}e \ + \ \bar{v}$$
\begin{boxA}
	\textbf{Sidenote:} \\
Since elements release electrons as they decay, and gain a proton, the element itself changes to the next element over in the periodic table.
\end{boxA}

\subsection{$\bm{\gamma}$-radiation}


\end{document}
