\documentclass[12pt]{article}
\usepackage{amsmath}
\usepackage{bm}
\usepackage{amssymb}
\usepackage{tikz}
\usepackage{tkz-euclide}
\usepackage{circuitikz}
\usepackage[many]{tcolorbox}

\newtcolorbox{boxA}{
colframe=white,
colback=white,
enhanced,
boxrule=1.5pt,
borderline = {0.75pt}{0pt}{dashed}
}

\tikzset{european}
\title{Nuclear Physics}
\author{Hertzberg, Joakim D.}
\date{\today}
\begin{document}
\begin{titlepage}
\maketitle
\begin{center}
\thispagestyle{empty}
\end{center}
\end{titlepage}

\tableofcontents

\newpage

\section{Particles and their properties}
\subsection{The atomic mass unit}

The atomic mass unit (also \emph{AMU}) is defined as $1.66053907 10^{-27} \ Kg$. It has the unit \emph{u}.
It was defined such that the mass of a proton (and a neutron, which has the same mass) is $1u$.

\subsection{The electron}

The electron is an \emph{elementary particle}, which is a particle that has no building blocks, but is itself among the smallest possible building blocks.
The mass of an electron is $0.000548579909 \ u$, and it has a charge of $1.60217663 \times 10^{-19} \ C$, which is equal to the elementary charge $e$.

\subsection{The proton}
The proton is a \emph{non-elementary} particle, which consists of \emph{quarks} which are elementary particles. It has a mass of $1 \ u$, and is positively charged by $+1 \ e$.

\subsection{The neutron}
The neutron is, as it's name implies, neutrally charged, meaning it has a charge of $0$, it does however have the same mass as a \emph{proton}, i.e. $1 \ u$.

\subsection{The positron}

The \emph{positron} is something which may also be encountered, and can easily be confused with the proton. It is significant to note that the positron is an \emph{elementary particle}, and is the anti-matter form of the \emph{electron}, so it has the same mass ($0.000548579909 \ u$), and a charge of $+1 \ e$.

\section{Types of radiation}




\subsection{$\bm{\alpha}$-radiation}
\emph{Alpha Radiation} or \emph{Alpha Decay} is when a particle releases another particle, a so-called \emph{$\alpha$-particle} as it decays.
$$^{A}_{Z}X \rightarrow \ \ ^{A-4}_{Z-2}X' + \ ^{4}_{2}\alpha$$

\begin{boxA}
	\textbf{\underline{What \emph{is} an $\bm{\alpha}$-particle?}} \bigbreak
	An alpha-particle looks exactly like a $He$-nucleus, that is, two protons \& 2 neutrons. Ignoring the amount of electrons, following is true: $$^{4}_{2}\alpha \ = \ ^{4}_{2}He$$
\end{boxA}


\subsection{$\bm{\beta}$-radiation}
\emph{Beta Radiation} or \emph{Beta Decay} is when a particle releases an electron and a neutrino as it decays, by a neutron splitting into an electron and a proton. 
\bigbreak
$$^{A}_{Z}X \rightarrow \ \ ^{A}_{Z+1}X' \ + \ ^{\ 0}_{-1}e \ + \ \bar{v}$$
\begin{boxA}
	\textbf{\underline{Sidenote:}} \bigbreak
Since elements release electrons as they decay, and gain a proton, the element itself changes to the next element over in the periodic table.
\end{boxA}

\subsection{$\bm{\gamma}$-radiation}

\newpage

\section{Nuclear Reactions}
\subsection{Transmutation}
It is possible to transmute one particle into the other by fusion. For example: $$^{14}_{\ 7}N \ + \ ^{4}_{2}He \ \rightarrow \ \ ^{17}_{\ 8}O \ + \ ^{1}_{1}H$$



\end{document}
